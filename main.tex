\documentclass[12pt]{article} 
\usepackage[a4paper, margin=1in]{geometry} 
\usepackage{graphicx} 
\usepackage{amsmath, amssymb, amsthm, bm} 
\usepackage{mathtools} 
\usepackage[numbers,sort&compress]{natbib} 
\usepackage{hyperref} 
\usepackage{titlesec} 
\usepackage{microtype} 
\usepackage{tikz-cd} 
\usepackage{float}
\usepackage[toc,page]{appendix}
\usepackage{enumitem}
\usepackage{booktabs} 
\usetikzlibrary{arrows.meta,decorations.pathreplacing}

\numberwithin{equation}{section} 
\hypersetup{
    colorlinks=true, 
    linkcolor=blue, 
    citecolor=magenta, 
    urlcolor=teal
}

% Title formatting
\titleformat{\section}{\Large\bfseries}{\thesection}{1em}{}
\titleformat{\subsection}{\large\bfseries}{\thesubsection}{1em}{}
\titleformat{\subsubsection}{\normalsize\bfseries}{\thesubsubsection}{1em}{}

\title{\textbf{Quantum Algorithms in Finance: A Comprehensive Literature Review}}
\author{Ehsan Massah}
\date{October 2025}

\begin{document}

\maketitle

\begin{abstract}
This literature review examines the emerging field of quantum and quantum-inspired algorithms in financial applications, with particular emphasis on portfolio optimization, risk management, derivatives pricing, and fraud detection. We survey the theoretical foundations of quantum computing approaches including quantum annealing, variational quantum eigensolvers (VQE), and quantum approximate optimization algorithms (QAOA), as well as quantum-inspired classical algorithms based on tensor networks and Ising formulations. Through analysis of recent empirical studies and benchmarking results, we assess the current state of quantum advantage in finance, identify key limitations and challenges, and outline promising directions for future research. Our review suggests that while full quantum advantage remains elusive, quantum-inspired methods show practical promise for discrete optimization problems, and hybrid classical-quantum approaches may offer the most realistic near-term pathway to commercial deployment.
\end{abstract}

\newpage
\tableofcontents
\newpage

%===========================================
\section{Introduction}
%===========================================

The intersection of quantum computing and finance represents one of the most promising frontiers for practical quantum advantage. Financial institutions face computational challenges that span portfolio optimization, risk assessment, derivatives pricing, and fraud detection—many of which involve NP-hard combinatorial problems or high-dimensional Monte Carlo simulations \citep{Rebentrost2018QuantumFinance, Egger2020QuantumFinance}.

Quantum computing offers potential speedups through three key paradigms: quantum annealing for optimization, gate-based quantum algorithms for specific computational tasks, and quantum machine learning for pattern recognition. In parallel, quantum-inspired (QI) algorithms have emerged as classical methods that adopt quantum-inspired structures without requiring quantum hardware \citep{Orus2019, Arrazola2019QI}.

This review is organized as follows: Section~\ref{sec:foundations} establishes the theoretical foundations of quantum computing in finance. Section~\ref{sec:portfolio_opt} examines portfolio optimization and risk management applications. Section~\ref{sec:pricing} discusses derivatives pricing and Monte Carlo methods. Section~\ref{sec:qi_algorithms} analyzes quantum-inspired classical algorithms. Section~\ref{sec:ml_fraud} covers quantum machine learning and fraud detection. Section~\ref{sec:benchmarks} presents empirical benchmarks and performance comparisons. Finally, Section~\ref{sec:challenges} identifies challenges and future research directions.

%===========================================
\section{Theoretical Foundations}
\label{sec:foundations}
%===========================================

\subsection{Quantum Computing Paradigms}

\subsubsection{Quantum Annealing}

Quantum annealing (QA) is an optimization method that exploits quantum tunneling to escape local minima in complex energy landscapes. D-Wave Systems pioneered commercial quantum annealers, which solve problems formulated as Ising models or Quadratic Unconstrained Binary Optimization (QUBO) problems \citep{Kadowaki1998QA}.

The Ising Hamiltonian is expressed as:
\begin{equation}
H_{\text{Ising}} = -\sum_{i<j} J_{ij} s_i s_j - \sum_i h_i s_i,
\end{equation}
where $s_i \in \{-1, +1\}$ are spin variables, $J_{ij}$ represent interaction strengths, and $h_i$ are local fields. The equivalent QUBO formulation uses binary variables $x_i \in \{0,1\}$:
\begin{equation}
H_{\text{QUBO}} = \sum_{i \leq j} Q_{ij} x_i x_j,
\end{equation}
where $Q$ is the QUBO matrix encoding the objective and constraints.

\subsubsection{Gate-Based Quantum Computing}

Gate-based quantum computers use quantum gates to manipulate qubits, enabling algorithms like Grover's search and Shor's factoring algorithm. In finance, gate-based approaches include:

\begin{itemize}[leftmargin=*]
\item \textbf{Quantum Approximate Optimization Algorithm (QAOA)}: A hybrid quantum-classical algorithm that prepares parameterized quantum states to approximate solutions to combinatorial optimization problems \citep{Farhi2014QAOA}.

\item \textbf{Variational Quantum Eigensolver (VQE)}: Uses parameterized quantum circuits to find the ground state of a Hamiltonian, applicable to portfolio optimization when formulated as an eigenvalue problem \citep{Peruzzo2014VQE}.

\item \textbf{Quantum Amplitude Estimation (QAE)}: Provides quadratic speedup over classical Monte Carlo for option pricing and risk assessment \citep{Brassard2002QAE, Rebentrost2018QuantumFinance}.
\end{itemize}

\subsection{Complexity and Quantum Advantage}

The concept of \textit{quantum advantage} (or quantum supremacy) refers to quantum computers solving problems intractable for classical computers. In finance, potential quantum advantages include:

\begin{itemize}[leftmargin=*]
\item \textbf{Quadratic speedup} in Monte Carlo methods via amplitude estimation
\item \textbf{Exponential speedup} for certain linear algebra problems (HHL algorithm) \citep{Harrow2009HHL}
\item \textbf{Improved optimization} through quantum tunneling and superposition
\end{itemize}

However, practical quantum advantage requires error-corrected, large-scale quantum computers—a threshold not yet achieved. Current Noisy Intermediate-Scale Quantum (NISQ) devices have limited qubits and high error rates, restricting their practical applicability \citep{Preskill2018NISQ}.

%===========================================
\section{Portfolio Optimization and Risk Management}
\label{sec:portfolio_opt}
%===========================================

\subsection{Classical Portfolio Theory}

Modern portfolio theory, pioneered by Markowitz \citep{Markowitz1952}, formulates portfolio selection as:
\begin{equation}
\min_{w \in \mathbb{R}^N} \; w^\top \Sigma w \quad \text{subject to} \quad \mu^\top w \geq r_{\min}, \; \mathbf{1}^\top w = 1, \; w \geq 0,
\label{eq:markowitz}
\end{equation}
where $w$ is the portfolio weight vector, $\Sigma$ is the covariance matrix, $\mu$ is expected returns, and $r_{\min}$ is the minimum required return.

When additional constraints are imposed—such as cardinality limits (restricting the number of assets), minimum holding sizes, sector exposure limits, or transaction costs—the problem becomes NP-hard and requires mixed-integer programming (MIQP) or heuristic methods \citep{Chang2000Cardinality}.

\subsection{QUBO Formulations for Portfolio Selection}

A significant body of research reformulates constrained portfolio optimization into QUBO form, making it amenable to quantum annealing or quantum-inspired optimization.

\citet{Elsokkary2017FinancialPortfolio} demonstrated early QUBO encodings for the portfolio optimization problem on D-Wave quantum annealers. Their approach discretizes portfolio weights into binary decision variables and incorporates budget and cardinality constraints through penalty terms. The general QUBO objective becomes:
\begin{equation}
\min_{x \in \{0,1\}^N} \; x^\top Q x + \lambda_1 P_{\text{budget}}(x) + \lambda_2 P_{\text{cardinality}}(x) + \lambda_3 P_{\text{return}}(x),
\label{eq:qubo_portfolio}
\end{equation}
where $Q$ encodes the risk (covariance), and $P_{\text{constraint}}$ terms enforce constraints via penalty functions with calibration parameters $\lambda_i$.

\citet{QIPQUBO2024} provide a comprehensive analysis of penalty calibration in QUBO portfolio formulations using real market data over a ten-year period. They demonstrate that poorly scaled penalties distort the energy landscape, leading to infeasible or suboptimal solutions. Their findings emphasize the critical importance of systematic penalty tuning, a challenge echoed in quantum annealing implementations \citep{Rebentrost2018QuantumFinance}.

Commercial quantum annealing systems—such as the Fujitsu Digital Annealer and Toshiba Simulated Bifurcation Machine—have been applied to high-quality liquid asset (HQLA) management, balance sheet optimization, and credit portfolio construction. These systems solve large-scale Ising problems in sub-second timeframes, suggesting potential for low-latency risk management \citep{Venturelli2019TemporalQA}.

\subsection{Variational Quantum Algorithms for Portfolio Optimization}

Beyond quantum annealing, gate-based variational algorithms like QAOA and VQE have been applied to portfolio optimization. \citet{Brandhofer2023QAOA} compare QAOA's performance on portfolio problems against classical heuristics and quantum annealing, finding that QAOA can match or exceed quantum annealing quality on small instances but suffers from optimization landscape challenges as problem size increases.

\citet{Phillipson2020Portfolio} explore portfolio rebalancing using VQE on IBM quantum hardware, demonstrating proof-of-concept results but acknowledging that circuit depth and gate errors limit practical scalability on current NISQ devices.

\subsection{Risk Metrics Beyond Variance}

Most quantum portfolio studies focus on variance minimization (mean-variance optimization). However, modern risk management emphasizes tail-risk metrics such as Conditional Value-at-Risk (CVaR) and Expected Shortfall (ES), particularly for regulatory compliance \citep{Rockafellar2000CVaR}.

\citet{Orus2019} outline theoretical extensions of QUBO formulations to CVaR optimization but note that scenario-based CVaR requires significantly more binary variables, increasing hardware requirements. \citet{Egger2020QuantumFinance} propose quantum algorithms for risk analytics using amplitude estimation but highlight that realistic financial scenarios demand error-corrected quantum computers beyond current capabilities.

Recent work by \citet{Woerner2019QAEOptions} demonstrates quantum speedups for CVaR estimation in option portfolios through quantum amplitude estimation, achieving quadratic speedup over Monte Carlo methods in simulation. However, experimental validation on real quantum hardware remains limited due to error rates and qubit constraints.

%===========================================
\section{Derivatives Pricing and Monte Carlo Methods}
\label{sec:pricing}
%===========================================

\subsection{Quantum Monte Carlo for Option Pricing}

Derivatives pricing often relies on Monte Carlo simulation to estimate expected payoffs under stochastic asset dynamics. Classical Monte Carlo converges as $O(1/\sqrt{M})$ where $M$ is the number of samples. Quantum amplitude estimation offers a quadratic speedup, converging as $O(1/M)$ \citep{Brassard2002QAE}.

\citet{Rebentrost2018QuantumFinance} formalize quantum algorithms for option pricing, demonstrating how quantum state preparation encodes probability distributions and amplitude estimation extracts expected values. Their framework applies to European, Asian, and barrier options under Black-Scholes dynamics.

\citet{Stamatopoulos2020QAE} implement quantum amplitude estimation for option pricing on IBM quantum hardware, pricing European call options with proof-of-concept results. They report that while the quantum algorithm achieves faster convergence theoretically, current hardware noise and limited qubits prevent practical advantage over classical methods.

\subsection{Credit Risk and Counterparty Exposure}

Credit risk modeling involves calculating probability distributions of default and exposure. \citet{Egger2020QuantumFinance} propose quantum algorithms for credit risk analytics, including:

\begin{itemize}[leftmargin=*]
\item Credit Valuation Adjustment (CVA) estimation via quantum amplitude estimation
\item Portfolio credit risk using quantum sampling methods
\item Systemic risk measures through quantum circuit simulations
\end{itemize}

Their results indicate potential speedups for CVA calculations, particularly in high-dimensional exposure scenarios. However, implementing realistic credit models on quantum hardware requires handling complex correlations and tail dependencies, which remains challenging with current gate fidelities.

\subsection{Path-Dependent Derivatives}

Path-dependent options (e.g., Asian options, lookback options) require tracking the entire price trajectory, making them computationally intensive. \citet{Woerner2019QAEOptions} extend quantum amplitude estimation to path-dependent payoffs, showing that quantum methods maintain their quadratic advantage even for complex path structures.

\citet{An2021QuantumLinear} explore quantum linear system algorithms (based on the HHL algorithm) for pricing path-dependent derivatives by solving associated partial differential equations (PDEs). While theoretically promising exponential speedup, HHL requires fault-tolerant quantum computers with thousands of logical qubits—far beyond current capabilities.

%===========================================
\section{Quantum-Inspired Classical Algorithms}
\label{sec:qi_algorithms}
%===========================================

\subsection{Tensor Networks for Optimization}

Quantum-inspired algorithms adopt structures from quantum mechanics—such as tensor networks (TNs)—to solve classical problems efficiently without quantum hardware. TNs originated in quantum many-body physics to compress high-dimensional quantum states using factorizations like Matrix Product States (MPS) and Tensor Trains (TT) \citep{Orus2014TensorNetworks}.

\citet{Mugel2020DynamicPortfolio} pioneer the application of tensor networks to \textit{Dynamic Portfolio Optimization}, formulating multi-period rebalancing with transaction costs as a TN optimization problem. Using eight years of real asset data, they demonstrate that TN-based solvers achieve competitive performance against mixed-integer programming while scaling more favorably.

A follow-up study published in \textit{Physical Review Research} \citep{Mugel2022Tensor} provides rigorous benchmarking, confirming that quantum-inspired TNs outperform D-Wave quantum annealers on realistic portfolio instances while maintaining solution quality comparable to state-of-the-art classical heuristics.

Further developments extend TN methods to industrial optimization contexts beyond finance, including energy systems, logistics, and large-scale scheduling \citep{QITNIndustrial2024, TNOpt2025}. The efficiency of TN compression depends critically on the correlation structure of the covariance matrix: when correlations exhibit low effective rank or local interactions, TNs provide significant computational advantages. However, dense or ill-conditioned covariance matrices reduce compression efficiency.

\subsection{Simulated Bifurcation and Digital Annealing}

Simulated bifurcation (SB) is a quantum-inspired optimization method that simulates the bifurcation dynamics of nonlinear oscillators to solve Ising problems. \citet{Goto2019SB} demonstrate that SB machines—such as the Toshiba Simulated Bifurcation Machine—can solve large-scale combinatorial problems orders of magnitude faster than simulated annealing.

\citet{Toshiba2021Finance} apply simulated bifurcation to portfolio optimization, balance sheet management, and liquidity optimization problems in banking, reporting sub-second solution times for problems with thousands of variables. Similarly, Fujitsu's Digital Annealer uses quantum-inspired parallel tempering to solve QUBO problems, with applications in portfolio optimization and asset allocation \citep{Aramon2019Fujitsu}.

These commercial QI platforms demonstrate practical advantages over classical MIQP solvers in specific problem regimes—particularly for discrete, highly constrained optimization with tight latency requirements.

\subsection{Hybrid Quantum-Inspired Frameworks}

Hybrid approaches combine quantum-inspired optimization with machine learning or classical heuristics to improve solution quality and robustness. \citet{TNGEO2023} propose TN-GEO (Tensor Network Generative Evolutionary Optimization), integrating tensor networks with generative models to explore solution spaces more efficiently. When applied to constrained portfolio optimization, TN-GEO achieves performance competitive with advanced metaheuristics while maintaining interpretability.

\citet{HybridSurvey2024} survey hybrid classical-quantum architectures, outlining frameworks that embed QI modules for constraint handling, warm-start initialization, or local search acceleration. Such hybridization represents a pragmatic pathway to near-term practical advantage, leveraging QI speed without requiring fault-tolerant quantum processors.

%===========================================
\section{Quantum Machine Learning and Fraud Detection}
\label{sec:ml_fraud}
%===========================================

\subsection{Quantum Machine Learning Foundations}

Quantum machine learning (QML) aims to leverage quantum computing for pattern recognition, classification, and clustering tasks. Key QML approaches include:

\begin{itemize}[leftmargin=*]
\item \textbf{Quantum Support Vector Machines (QSVM)}: Use quantum kernel estimation to classify high-dimensional data with potential exponential speedup \citep{Rebentrost2014QSVM}.
\item \textbf{Quantum Neural Networks (QNN)}: Parameterized quantum circuits trained via classical optimization to perform supervised learning \citep{Farhi2018QNN}.
\item \textbf{Quantum Generative Models}: Quantum circuits that learn probability distributions, applicable to synthetic data generation and anomaly detection \citep{Benedetti2019QGAN}.
\end{itemize}

\subsection{Fraud Detection Applications}

Fraud detection in financial transactions involves identifying anomalies in high-dimensional transaction data. \citet{Liu2022QMLFraud} apply quantum machine learning to credit card fraud detection, using QSVMs to classify fraudulent transactions. Their results on benchmark datasets show accuracy competitive with classical methods but with current hardware limitations preventing scalability.

\citet{Kyriienko2021QuantumAnomalyDetection} propose quantum anomaly detection algorithms based on quantum autoencoders and amplitude estimation, demonstrating potential advantages for detecting rare events in imbalanced datasets. However, encoding classical transaction data into quantum states remains a bottleneck, often negating theoretical speedups.

\subsection{Market Prediction and Algorithmic Trading}

Several studies explore QML for financial time series prediction and algorithmic trading. \citet{Chen2023QuantumTrading} implement quantum reinforcement learning for portfolio management, training quantum agents to optimize trading strategies. While proof-of-concept results are promising, the authors acknowledge that classical deep reinforcement learning currently outperforms QML due to hardware constraints and training complexity.

\citet{Orus2019} note that quantum advantage in machine learning remains highly contested, as classical algorithms continue to improve and quantum hardware lacks the scale and error correction needed for practical deployment.

%===========================================
\section{Empirical Benchmarks and Performance Comparison}
\label{sec:benchmarks}
%===========================================

\subsection{Large-Scale Benchmarking Studies}

Rigorous benchmarking is essential to assess whether quantum and quantum-inspired algorithms provide practical advantages over classical methods. \citet{Stopfer2025Benchmark} conduct a comprehensive benchmark of quantum, quantum-inspired, and classical MIQP solvers on 250 real-world portfolio optimization instances. Their key findings include:

\begin{itemize}[leftmargin=*]
\item Optimized classical solvers (Gurobi, CPLEX) generally outperform both quantum annealers and QI algorithms on large, dense problem instances
\item QI methods excel in constrained or latency-sensitive scenarios where discrete structures dominate
\item Quantum annealing (D-Wave) shows promise for smaller problems but suffers from embedding overhead and limited connectivity
\item Benchmark fairness requires identical objectives, constraints, and data splits—often lacking in published comparisons
\end{itemize}

\citet{Mugel2022Tensor} provide comparative analysis of tensor network solvers versus quantum annealers and classical heuristics, confirming that TNs offer scalability advantages but only modest risk-adjusted performance improvements over tuned classical methods.

\subsection{Hardware Limitations and Error Rates}

Current quantum hardware faces significant limitations:

\begin{itemize}[leftmargin=*]
\item \textbf{Qubit count}: NISQ devices have 50-100 qubits; financial problems often require thousands
\item \textbf{Gate fidelity}: Error rates of 0.1-1\% per gate limit circuit depth
\item \textbf{Connectivity}: Limited qubit connectivity requires costly SWAP operations
\item \textbf{Coherence time}: Quantum states decohere within microseconds, constraining algorithm execution
\end{itemize}

\citet{Preskill2018NISQ} argue that demonstrating practical quantum advantage requires algorithms specifically designed for NISQ constraints, motivating the development of variational and hybrid approaches.

\subsection{Cost-Benefit Analysis}

\citet{Egger2020QuantumFinance} perform a cost-benefit analysis of quantum computing investments in finance, concluding that:

\begin{itemize}[leftmargin=*]
\item Current quantum hardware does not provide cost-effective advantages over classical HPC infrastructure
\item Quantum advantage is most likely in niche applications with exponential classical complexity
\item Financial institutions should invest in algorithm development and hybrid frameworks to prepare for future quantum hardware improvements
\end{itemize}

%===========================================
\section{Challenges and Future Research Directions}
\label{sec:challenges}
%===========================================

\subsection{Current Limitations}

The surveyed literature identifies several critical limitations:

\begin{enumerate}[label=\arabic*., leftmargin=*]
\item \textbf{Hardware Maturity}: Quantum computers remain in the NISQ era, with insufficient qubits, high error rates, and limited connectivity for practical financial applications.

\item \textbf{Encoding Overhead}: Converting classical financial data into quantum states incurs computational costs that can negate theoretical speedups.

\item \textbf{Penalty Calibration}: QUBO formulations require careful tuning of constraint penalties; poor calibration leads to infeasible or suboptimal solutions \citep{QIPQUBO2024}.

\item \textbf{Scalability Under Dense Correlations}: Tensor networks and Ising models lose compression efficiency when covariance matrices are dense or ill-conditioned \citep{QITNIndustrial2024}.

\item \textbf{Nonstationarity}: Most quantum studies assume stationary covariance structures, neglecting regime shifts and time-varying correlations prevalent in financial markets.

\item \textbf{Limited Risk Metrics}: The majority of research targets variance minimization; extensions to tail-risk metrics (CVaR, ES) remain underdeveloped despite regulatory importance.

\item \textbf{Benchmark Inconsistency}: Performance comparisons often lack identical objectives, constraints, and data partitions, hindering fair evaluation \citep{Stopfer2025Benchmark}.
\end{enumerate}

\subsection{Promising Research Directions}

Despite these challenges, several promising avenues emerge:

\begin{itemize}[leftmargin=*]
\item \textbf{Tail-Risk Optimization}: Developing QUBO and TN encodings for CVaR and Expected Shortfall, incorporating scenario-based stress testing frameworks.

\item \textbf{Adaptive Quantum-Inspired Methods}: Algorithms capable of dynamically updating penalties or tensor ranks in response to changing market regimes and correlation structures.

\item \textbf{Hierarchical Decomposition}: Multi-level frameworks that cluster assets or sectors first, then solve local subproblems via QI optimization, reducing problem size and improving scalability.

\item \textbf{Hybrid Quantum-Classical Pipelines}: Using QI methods for warm-start initialization in QAOA or VQE, or embedding quantum subroutines within classical optimization loops \citep{Rebentrost2018QuantumFinance}.

\item \textbf{Error Mitigation Techniques}: Developing robust error mitigation strategies for NISQ devices to improve solution quality without requiring full error correction.

\item \textbf{Explainable Quantum Finance}: Designing interpretable quantum and QI frameworks that map portfolio weights and constraints directly to risk factor sensitivities, enhancing regulatory compliance and transparency.

\item \textbf{Real-Time Risk Management}: Leveraging the speed of quantum-inspired digital annealers and simulated bifurcation machines for intraday risk monitoring and high-frequency rebalancing.

\item \textbf{Quantum-Enhanced Machine Learning}: Advancing QML algorithms for fraud detection, market prediction, and sentiment analysis as quantum hardware improves.
\end{itemize}

%===========================================
\section{Conclusion}
%===========================================

Quantum and quantum-inspired algorithms represent a frontier in computational finance, offering novel approaches to optimization, risk management, and machine learning challenges. While full quantum advantage remains elusive due to hardware limitations, quantum-inspired classical algorithms—particularly tensor networks, simulated bifurcation, and QUBO-based methods—demonstrate practical promise for discrete, constrained optimization problems.

Current benchmarks indicate that optimized classical solvers remain competitive or superior for most financial applications. However, QI methods excel in latency-sensitive environments and problems with complex combinatorial constraints. As quantum hardware matures, hybrid classical-quantum frameworks are likely to form the bridge between today's high-performance computing and tomorrow's fault-tolerant quantum processors.

Future research should prioritize tail-risk metrics, adaptive algorithms for nonstationary markets, hierarchical problem decomposition, and explainable quantum models. Financial institutions investing in quantum algorithm development, rather than premature hardware adoption, are best positioned to capture eventual quantum advantages.

In summary, quantum algorithms in finance are transitioning from theoretical curiosity to practical experimentation. While significant challenges remain, the field's rapid progress suggests that quantum and quantum-inspired methods will increasingly complement and, in specific niches, surpass classical approaches in the coming decade.

\bibliographystyle{apalike}
\bibliography{quantum_inspired_risk}


\end{document}
